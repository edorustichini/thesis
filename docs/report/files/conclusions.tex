\chapter{Conclusioni}\label{ch:conclusioni}
Questo lavoro di tesi si è concentrato sulla verifica dell'efficacia della codifica di JPEG AI nel problema di identificazione di immagini generate sinteticamente. Sebbene siano già presenti diversi esempi di \textit{learning-based codec}, JPEG AI rappresenta il primo standard ufficiale (ISO/IEC 6048-1:2025) di compressione di questo tipo. Le sue caratteristiche lo rendono un possibile candidato per l'utilizzo su larga scala, motivando la scelta di concentrare il lavoro su questo.\\
I risultati ottenuti mostrano come le rappresentazioni latenti estratte dall'encoder JPEG AI siano sufficientemente ricche di informazioni, permettendo di raggiungere un'elevata accuratezza attraverso l'uso di modelli di classificazione basati su alberi di decisione.\\
L'analisi dei risultati ha evidenziato due principali fattori significativi per le prestazioni dei modelli: la scelta delle componenti di colore utilizzate per l'addestramento e il metodo di preprocessing delle feature. I modelli addestrati su tutte le componenti YUV hanno superato nella maggior parte dei casi i modelli allenati esclusivamente sulla componente di luminanza, indicando così che le informazioni contenute nelle componenti relative alla crominanza siano rilevanti per il tipo di problema affrontato. Inoltre, il metodo di flattening delle feature ha permesso di ottenere le migliori prestazioni, a discapito però di un aumento significativo dei tempi di addestramento e delle risorse computazionali necessarie.\\
