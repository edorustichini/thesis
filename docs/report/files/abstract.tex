\chapter*{Abstract}\label{ch:abstract}
\iffalse
I continui e rapidi progressi nel campo dell'intelligenza artificiale hanno portato ad avere tecniche per la generazione di contenuti digitali che risultano avere un livello di realismo tale da essere difficilmente distinguibili da quelli reali.
Questi contenuti vengono chiamati "\textit{deepfake}", ed il loro rilevamento è uno dei temi principali della \textit{multimedia forensics}, disciplina che consiste nello studio e analisi di contenuti digitali con il fine di verificarne l'autenticità.

La "Deepfake Dettion" è un argomento di grande interesse perché con le nuove tecnologie, nonostante siano state create per scopi principalmente di intrattenimento o comunque innocui, è estremamente facile anche per non esperti generare dei contenuti digitali falsi molto realistici. Questa possibilità, soprattutto in casi di immagini che rappresentano volti, può sfociare in seri rischi tra cui: diffusione di disinformazione, danni all'immagine di figure pubbliche e private.
\fi
Questo lavoro si propone di esaminare l'uso di JPEG AI, un codec di compressione per immagini, ultimamente diventato standard, basato su reti neurali per svolgere task di deepfake detecttion.
L'utilizzo del codec in questo contesto non è ancora stato esplorato, quindi l'obbiettivo è quello di verificare se le rappresentazioni latenti delle immagini estratte dall'encoder di JPEG AI siano sufficienti per rilevare se un'immagine è falsa o meno.
