\chapter*{Introduzione}\label{ch:intro}
I continui progressi nel campo dell'intelligenza artificiale hanno sviluppato tecniche per la generazione di contenuti digitali con un livello di realismo tale da renderli indistinguibili da quelli reali. La diffusione di queste tecnologie al grande pubblico aumenta il rischio di utilizzi illeciti, con potenziali conseguenze negative. Il rilevamento di questo tipo di contenuti è uno dei temi principali della \textit{multimedia forensics}, disciplina che consiste nello studio e analisi di contenuti digitali con il fine di verificarne l'autenticità.\\
Questo lavoro di tesi triennale si propone di esaminare l'uso della codifica di JPEG AI, un codec di compressione basato su reti neurali, per svolgere task di \textit{fake detection} su immagini di volti; in particolare si vuole verificare se l'encoder di JPEG AI riesca ad estrarre dalle immagini rappresentazioni latenti sufficientemente ricche di informazioni da rilevare se un'immagine sia sintetica o reale.
%TODO: spiegare perchè è importante questo lavoro
\paragraph{Struttura relazione}
La relazione è strutturata come segue: nel capitolo \ref{ch:theory} vengono presentati gli aspetti teorici necessari per la comprensione del lavoro; nel capitolo \ref{ch:work} viene presentato la metodologia e gli aspetti più tecnici riguardanti il lavoro svolto, mentre nel capitolo \ref{ch:exp} sono esposti gli esperimenti svolti e i loro risultati. Infine nel capitolo \ref{ch:conclusioni} sono tratte le conclusioni.

