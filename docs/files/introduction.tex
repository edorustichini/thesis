\chapter*{Introduzione}\label{ch:intro}
I continui progressi nel campo dell'intelligenza artificiale hanno portato ad avere tecniche per la generazione di contenuti digitali che risultano avere un livello di realismo tale da essere difficilmente distinguibili da quelli reali.
Il rilevamento di questo tipo di contenuti è uno dei temi principali della \textit{multimedia forensics}, disciplina che consiste nello studio e analisi di contenuti digitali con il fine di verificarne l'autenticità.\\
Nonostante queste tecnologie abbiano applicazioni creative ed innovative in campi come quello dell'intrattenimento, ormai è estremamente facile anche per non esperti generare dei contenuti digitali falsi molto realistici, implicando molti rischi per possibili usi illeciti.\\
Questo lavoro di tesi triennale si propone di esaminare l'uso di JPEG AI, un codec di compressione basato su reti neurali, per svolgere task di \textit{fake detection} su immagini di volti; in particolare si vuole verificare se l'encoder di JPEG AI riesce ad estrarre dalle immagini delle rappresentazioni latenti sufficientemente ricche di informazioni da riuscire a rilevare se un'immagine sia sintetica utiluzzando un semplice classificatore.
%TODO: spiegare perchè è importante questo lavorp
\subsubsection*{Struttura relazione}
La relazione è strutturata come segue: nel capitolo \ref{ch:theory} vengono presentati gli aspetti teorici necessari per capire il resto del lavoro, tra cui la creazione e generazione di immagine sintetiche (sez. \ref{sec:fakeimg}) ed il codec JPEG AI sez. \ref{sec:jpegai}); nel capitolo \ref{ch:work} viene presentato la metodologia e gli aspetti più tecnici riguardanti il lavoro svolto, mentre nel capitolo \ref{ch:exp} sono spiegati gli esperimenti e i loro risultati, ed infine nel capitolo \ref{ch:conclusioni} sono tratte le conclusioni.

